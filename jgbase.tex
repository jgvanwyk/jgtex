% Load the article class, forwarding all provided options to it
\DeclareOption*{\PassOptionsToClass{\CurrentOption}{article}}
\ProcessOptions\relax
\LoadClass{article}

% Packages
\RequirePackage{xcolor}
\RequirePackage{hyperref}
\RequirePackage{amsthm}
\RequirePackage{amsmath}
\RequirePackage{newtxtext}
\RequirePackage{newtxmath}
\RequirePackage{enumitem}
\RequirePackage[nameinlink,capitalise]{cleveref}

% Styling
\setlist{%
    label         = (\alph*),
    listparindent = \parindent,
    topsep        = \smallskipamount,
    parsep        = \parskip,
    left          = .5ex}

\if@titlepage
    \let\articlemaketitle\maketitle
    \renewcommand{\maketitle}{\hypersetup{pageanchor=false}\articlemaketitle}
\fi

\AtBeginDocument{%
    \bigskipamount.7\baselineskip plus.7\baselineskip
    \medskipamount\bigskipamount \divide\medskipamount\tw@
    \smallskipamount\medskipamount \divide\smallskipamount\tw@
    \abovedisplayskip\medskipamount
    \belowdisplayskip \abovedisplayskip
    \abovedisplayshortskip\abovedisplayskip
    \advance\abovedisplayshortskip-1\abovedisplayskip
    \belowdisplayshortskip\abovedisplayshortskip
    \advance\belowdisplayshortskip 1\smallskipamount
    \jot\baselineskip \divide\jot 4 \relax}

\renewcommand{\@seccntformat}[1]{\csname the#1\endcsname\enspace}

\let\defn\emph
\newcommand{\comment}[1]{{\color{red}#1}}
\let\vec\mathbf

\renewcommand{\Phi}{\varPhi}
\renewcommand{\Gamma}{\varGamma}
\renewcommand{\Lambda}{\varLambda}
\renewcommand{\Omega}{\varOmega}
\renewcommand{\phi}{\varphi}
\renewcommand{\epsilon}{\varepsilon}
\renewcommand{\emptyset}{\varnothing}
\renewcommand{\implies}{\Rightarrow}
\renewcommand{\iff}{\Leftrightarrow}
\renewcommand{\cong}{\equiv}
\let\Re\@jgundefined
\let\Im\@jgundefined
\newcommand{\Re}{\mathoperator{Re}}
\newcommand{\Im}{\mathoperator{Im}}

\let\section\@jgundefined
\let\subsection\@jgundefined
\let\subsubsection\@jgundefined
\let\paragraph\@jgundefined
\let\subparagraph\@jgundefined

\newtheoremstyle{theorem}%
    {.5\baselineskip\@plus.2\baselineskip\@minus.2\baselineskip}%
    {.5\baselineskip\@plus.2\baselineskip\@minus.2\baselineskip}%
    {\itshape}%
    {0pt}%
    {\bfseries}%
    {.}%
    {0.5em}%
    {\thmnumber{#2\enspace}\thmname{#1}\thmnote{\ (#3)}}


\newtheoremstyle{definition}%
    {.5\baselineskip\@plus.2\baselineskip\@minus.2\baselineskip}%
    {.5\baselineskip\@plus.2\baselineskip\@minus.2\baselineskip}%
    {\normalfont}%
    {0pt}%
    {\bfseries}%
    {.}%
    {0.5em}%
    {\thmnumber{#2\enspace}\thmname{#1}\thmnote{\ (#3)}}

\numberwithin{equation}{\@jgnumberingcounter}

\crefname{equation}{}{}
\crefname{section}{Section}{Sections}
\crefname{subsection}{Subsection}{Subsections}

\theoremstyle{theorem}

\newtheorem{theorem}{Theorem}[\@jgnumberingcounter]
\newtheorem*{theorem*}{Theorem}
\crefname{theorem}{}{}

\newtheorem{lemma}[theorem]{Lemma}
\newtheorem*{lemma*}{Lemma}

\newtheorem{proposition}[theorem]{Proposition}
\newtheorem*{proposition*}{Proposition}
\crefname{proposition}{}{}

\newtheorem{corollary}[theorem]{Corollary}
\newtheorem*{corollary*}{Corollary}
\crefname{corollary}{}{}

\theoremstyle{definition}

\newtheorem{definition}[theorem]{Definition}
\newtheorem*{definition*}{Definition}
\crefname{definition}{}{}

\newtheorem{remark}[theorem]{Remark}
\newtheorem*{remark*}{Remark}
\crefname{remark}{}{}

\newtheorem{example}[theorem]{Example}
\newtheorem*{example*}{Example}
\crefname{example}{}{}